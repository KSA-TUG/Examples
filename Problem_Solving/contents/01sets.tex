% !TEX root = ./ProblemSolving.tex

\chapter{Sets}\label{chap:sets}

Set is the most important concept in mathematics.
When we define a something or prove a theorem, we usually use the concept of sets.
Thus the deep understanding of the concept of sets is a basic of mathematics.

%%%%%%%%%%%%%%%%%%%%%%%%%%%%%%%%%%%%%%%%%%%%%%%%%%%%%%%%%%%%%%%%%%%%%%%%%%%%%%%%%%%%%%%%%%%%
\section{Sets}
%%%%%%%%%%%%%%%%%%%%%%%%%%%%%%%%%%%%%%%%%%%%%%%%%%%%%%%%%%%%%%%%%%%%%%%%%%%%%%%%%%%%%%%%%%%%
%%%%%%%%%%%%%%%%%%%%%%%%%%%%%%%%%%%%%%%%%%%%%%%%%%%%%%%%%%%%%%%%%%%%%%%%%%%%%%%%%%%%%%%%%%%%

We think of a \term{set} as a collection of distinct objects with a precise description that
provides a way of deciding(in principle) whether a given objects is in it.

Let $x$ and $y$ be two objects.
We write $x=y$ when $x$ and $y$ are equal,
that is, when they are precisely the same object.
We write $x\neq y$ when $x$ and $y$ are not equal.

\begin{defn}
The objects in a set are its \term{elements} or \term{members}.
When $x$ is an elements of $A$, we write \term{$x\in A$} and say \term{$x$ belong to $A$}.
When $x$ is not in $A$, we write \term{$x\not\in A$}.
\end{defn}
%
%
A set can be specified by stating the common property of its objects.
In particular, we write
\[
A=\left\{x\,\vert\, P(x)\right\}
\]
when $A$ is the set of all objects $x$ that have the property $P$.

\begin{eg}
We can specify the following sets:
\begin{align*}
\text{Suits} & = \{x\, \vert\, \text{$x$ is a suit in an Anglo-French deck of cards}\}\\
\text{Even} & = \{x\, \vert\, \text{$x$ is an even natural number}\}\\
\text{Odd} & = \{x\, \vert\, \text{$x$ is an odd natural number}\}\\
\text{Prime} & = \{x\, \vert\, \text{$x$ is a prime natural number}\}
\end{align*}
\end{eg}
%
%
If the number of its objects is finite, a set can be specified by listing its objects
between braces. In particular, we write
\[
A = \{x_1, x_2, \ldots, x_n\}
\]
when $A$ contains the objects $x_1$, $\ldots$, $x_n$, and nothing else.

\begin{eg}
\begin{align*}
\text{Suits} & = \{\clubsuit, \diamondsuit, \heartsuit, \spadesuit \}
\end{align*}
\end{eg}
%
%
Sometimes, we may informally specify a set by listing some of its objects, and
then let the context help us ``figure out" the rest. In particular, we write
\[
A = \{x_1, x_2, x_3, \ldots\}
\]
when $A$ contains the objects $x_1$, $x_2$, $x_3$, as well as other objects more or less
determined by the context.
This method is convenient when the number of objects of the set is infinite.
It is also convenient when the number of objects of the set if finite but ``large".

\begin{eg}
\begin{align*}
\text{Even} & = \{0, 2, 4, \ldots\}\\
\text{Odd} & = \{1, 3, 5, \ldots\}
\end{align*}
\end{eg}

\TIP{경우에 따라서 또는 사정에 따라서 자연수의 집합에 $0$을 포함시키기도 하고 제외시키기도 한다.
이 책에서는 편의상 자연수의 집합에 $0$을 포함시키도록 하겠다.}
We use the symbols $\mathbb{N}$, $\mathbb{Z}$, $\mathbb{Q}$, $\mathbb{R}$, and $\mathbb{C}$
to denote the following sets of numbers:
\begin{itemize}
\item $\mathbb{N} = \{0, 1,2,3,\ldots\}$ is the set of natural numbers.
\item $\mathbb{Z} = \{\ldots, -2, -1, 0, 1, 2,\ldots\}$ is the set of integers.
\item $\mathbb{Q} = \{\frac{a}{b}\,\vert\, \text{$a$, $b\in\mathbb{Z}$ and $b\neq0$}\}$ is the set of rational numbers.
\item $\mathbb{R}$ is the set of real numbers.
\item $\mathbb{C} = \{a+ib\,\vert\, \text{$a$, $b\in\mathbb{R}$ and $i=\sqrt{-1}$}\}$
\end{itemize}


\begin{defn}
%The objects in a set are its \term{elements} or \term{members}.
%When $x$ is an element of $A$, we write $x\in A$ and say \term{$x$ belong to $A$}.
%When $x$ is not in $A$, we write $x\not\in A$.
Sets $A$ and $B$ are \term{equal}, written $A=B$,
if they have the same elements.
The \term{empty set}, written $\emptyset$, is the unique set with no elements.
\end{defn}


\begin{eg}
When we specify a set by listing its elements,
their order does not matter.
For instance, we have
\[
\text{Suits} = \{\clubsuit, \diamondsuit, \heartsuit, \spadesuit \} = \{\diamondsuit, \heartsuit, \clubsuit, \spadesuit \}
\]
And repetitions do not matter too.
For instance, we have
\[
\text{Suits} = \{\clubsuit, \diamondsuit, \heartsuit, \spadesuit \}
= \{\clubsuit, \clubsuit, \diamondsuit, \heartsuit, \spadesuit \}
\]
\end{eg}


\begin{eg}[The Penny Problem]
Given piles of pennies, we remove one coin from each pile to make one new pile.
Each original pile shrinks by one,
so each pile of size one disappears: $1,1,2,5$ becomes $1,4,4$, for example.
We consider different orderings of the same list of sizes to be equivalent,
so we restrict our attention to lists of positive integers in nondecreasing order.
Let $S$ be the set of list that do not change.
$$
\begin{tabular}{cccc}
 & & & $\bigcirc$\\
 & & & $\bigcirc$\\
 & & & $\bigcirc$\\
 & & $\bigcirc$ & $\bigcirc$\\
$\bigcirc$ & $\bigcirc$ & $\bigcirc$ & $\bigcirc$
\end{tabular}
\qquad \Longleftrightarrow \qquad
\begin{tabular}{ccc}
 & $\bigcirc$ & $\bigcirc$\\
 & $\bigcirc$ & $\bigcirc$\\
 & $\bigcirc$ & $\bigcirc$\\
$\bigcirc$ & $\bigcirc$ & $\bigcirc$
\end{tabular}
$$
Let $a$ be a list with $n$ piles,
and let $b$ the resulting new list.
If $a\in S$, meaning that $a$ and $b$ are the same, then $b$ also has $n$ piles.
Since we introduce one new pile, exactly one pile must disappear.
Thus $a$ has exactly one pile of size $1$.
Thus $b$ also has exactly one pile of size $1$.
This forces $a$ to have exactly one pile of size $2$.

We continue this reasoning for $i$ from $1$ to $n-1$.
From $a$ having one pile of size $i$, we conclude that $b$ has one pile of size $i$,
and therefore that $a$ has one pile of size $i+1$.
This gives us one pile of size $1$ through $n$.

Let $T$ be the set of lists consisting of one pile of each size from $1$ through some natural number $n$.
We have shown that every unchanged configuration has this from, so $S\subseteq T$.
To complete the solution,
we also check that all elements of $T$ remain unchanged.

Consider the element of $T$ with piles of sizes $1,2,\ldots,n$.
For each $i$ from $2$ to $n$, the pile of size $i$ becomes a pile of size $i-1$.
The pile of size $1$ disappears, and the $n$ piles each contribute one coin to from a new pile of size $n$.
The result the original list.
Now we have proved that $S\subseteq T$ and $T\subseteq S$, so $S=T$.
We have described all the unchanged lists.
\end{eg}


%%%%%%%%%%%%%%%%%%%%%%%%%%%%%%%%%%%%%%%%%%%%%%%%%%%%%%%%%%%%%%%%%%%%%%%%%%%%%%%%%%%%%%%%%%%%
\section{Subsets}
%%%%%%%%%%%%%%%%%%%%%%%%%%%%%%%%%%%%%%%%%%%%%%%%%%%%%%%%%%%%%%%%%%%%%%%%%%%%%%%%%%%%%%%%%%%%
%%%%%%%%%%%%%%%%%%%%%%%%%%%%%%%%%%%%%%%%%%%%%%%%%%%%%%%%%%%%%%%%%%%%%%%%%%%%%%%%%%%%%%%%%%%%

\TIP{\centerline{For any set $A$, $\emptyset\subseteq A$.}}
\begin{defn}
Let $A$ and $B$ be sets.
We say that $A$ is a \term{subset} of $B$, written $A\subseteq B$,
if all elements of $A$ are also elements of $B$.
And $B$ is a \term{proper subset} of a set $A$, written $A\subset B$,
if $B$ is a subset of $A$ that is not $A$ itself.
\end{defn}


\begin{prop}
Let $A$, $B$, and $C$ be sets.
Then the following properties hold.
\begin{myEnum}
\item $A\subseteq A$
\item If $A\subseteq B$ and $B\subseteq A$, then $A=B$.
\item If $A\subseteq B$ and $B\subseteq C$, then $A\subseteq C$.
\end{myEnum}
\end{prop}


\TIP{\centerline{$\mathcal{P}(A)=\{B\,|\,B\subseteq A\}$}}
\begin{defn}
The \term{power set} of a set $A$, denoted by $\mathcal{P}(A)$, is the set of all subsets of $A$.
\end{defn}


\begin{eg}
Let $x\neq y$ and $A=\{x,y\}$. Then
\[
\mathcal{P}(A)=\left\{\emptyset,\{x\}, \{y\}, \{x,y\}\right\}
\]
\end{eg}



%%%%%%%%%%%%%%%%%%%%%%%%%%%%%%%%%%%%%%%%%%%%%%%%%%%%%%%%%%%%%%%%%%%%%%%%%%%%%%%%%%%%%%%%%%%%
\section{Set Operators}
%%%%%%%%%%%%%%%%%%%%%%%%%%%%%%%%%%%%%%%%%%%%%%%%%%%%%%%%%%%%%%%%%%%%%%%%%%%%%%%%%%%%%%%%%%%%
%%%%%%%%%%%%%%%%%%%%%%%%%%%%%%%%%%%%%%%%%%%%%%%%%%%%%%%%%%%%%%%%%%%%%%%%%%%%%%%%%%%%%%%%%%%%

\begin{defn}
Let $A$ and $B$ be sets.
Their \term{union}, written $A\cup B$, consists of all elements in $A$ or $B$.
Their \term{intersection}, written $A\cap B$, consists of all elements in both $A$ and $B$.
Their \term{difference}, written $A-B$, consists of all elements of $A$ that are not in $B$.
Two sets are \term{disjoint} if there intersection is the empty set $\emptyset$.
If a set $A$ is contained in some universe $U$ under discussion,
then the \term{complement} $A^c$ of $A$ is the set of elements of $U$ \myemph{not} in $A$.
\end{defn}


\begin{eg}
Let $E$ and $O$ denote the sets of even numbers and odd numbers.
We have $E\cap O=\emptyset$ and $E\cup O = \mathbb{Z}$.
Within $\mathbb{Z}$, we have $E^c=O$.
\end{eg}


\begin{defn}
In a \term{Venn Diagram}, an outer box represents the universe under consideration,
and regions within the box correspond to sets.
Non-overlaping regions corresponding to disjoint sets.
\end{defn}


\begin{eg}
\mPic{venn.png}
The four regions in the Venn diagram for two sets $A$ and $B$ represent $A\cap B$,
$(A\cup B)^c$, $A-B$, and $B-A$.
\end{eg}


\begin{prop}
\TIP{``P if and only if Q"라는 표현은 P라는 명제와 Q라는 명제가 동치(equivalent)라는 뜻으로 $\Leftrightarrow$을 사용하여 나타내기도 한다.

이 문장을 증명하기 위해서는 ``P이면 Q이다"라는 문장과 ``Q이면 P이다"라는 문장 두개를 증명하면 된다.
이와 관련된 자세한 내용은 이후 Logic 파트에서 알아보겠다.}
Let $A$ and $B$ be sets.
Then $A\subseteq B$ if and only if $A\cup B = B$.
\end{prop}
\begin{proof}
($\Rightarrow$) Suppose that $A\subseteq B$.
Then clearly $B\subseteq A\cup B$.
On the other hand, if $x\in A\cup B$, then either $x\in A$ or $x\in B$,
and in both cases we have $x\in B$.

($\Rightarrow$)
Suppose that $A\cup B = B$.
Let $x\in A$.
Then $x\in A\cup B$, which implies $x\in B$.
\end{proof}


\begin{prop}
\TIP{The set operator $\cup$ is associative, we can conveniently
drop the parenthesis when writing set expressions like $A\cup B\cup C$.}
Let $A$, $B$, and $C$ be sets.
Then
\begin{myEnum}
\item $A\cup A = A$\hfill\myem{(idempotency)}
\item $A\cup B = B\cup A$\hfill\myem{(commutativity)}
\item $(A\cup B)\cup C = A\cup(B\cup C)$\hfill\myem{(associativity)}
\end{myEnum}
\end{prop}
\begin{proof}
\myem{1.}
\begin{align*}
x\in A\cup A &\Leftrightarrow x\in A\ \text{or}\ x\in A\\
&\Leftrightarrow x\in A.
\end{align*}
\myem{2.}
\begin{align*}
x\in A\cup B &\Leftrightarrow x\in A\ \text{or}\ x\in B\\
&\Leftrightarrow  x\in B\ \text{or}\ x\in A\\
&\Leftrightarrow  x\in B \cup A.
\end{align*}
\myem{3.}
\begin{align*}
x\in (A\cup B)\cup C &\Leftrightarrow x\in A\cup B\ \text{or}\ x\in C\\
&\Leftrightarrow  x\in A\ \text{or}\ x\in B\ \text{or}\ x\in C\\
&\Leftrightarrow  x\in A\ \text{or}\ x\in B\cup C\\
&\Leftrightarrow  x\in A\cup(B \cup C).
\end{align*}
\end{proof}


\begin{hw}
Let $A$ and $B$ be sets.
Show that $A\subseteq B$ if and only if $A\cap B=A$.
\end{hw}


\begin{hw}
\TIP{The set operator $\cap$ is associative, we can conveniently
drop the parenthesis when writing set expressions like $A\cap B\cap C$.}
Let $A$, $B$, and $C$ be sets.
Then prove the following properties.
\begin{myEnum}
\item $A\cap A = A$\hfill\myem{(idempotency)}
\item $A\cap B = B\cap A$\hfill\myem{(commutativity)}
\item $(A\cap B)\cap C = A\cap(B\cap C)$\hfill\myem{(associativity)}
\end{myEnum}
\end{hw}



\begin{prop}
Let $A$, $B$, and $C$ be sets.
Then
\begin{myEnum}
\item $A\cup(A\cap B) = A$\hfill\myem{(absorption of $\cup$ over $\cap$)}
\item $A\cap(A\cup B) = A$\hfill\myem{(absorption of $\cap$ over $\cup$)}
\item $A\cup(B\cap C) = (A\cup B)\cap (A\cup C)$\hfill\myem{(distributivity of $\cup$ over $\cap$)}
\item $A\cap(B\cup C) = (A\cap B)\cup (A\cap C)$\hfill\myem{(distributivity of $\cap$ over $\cup$)}
\end{myEnum}
\end{prop}
\begin{proof}
\myem{1.}
Let $x\in A\cup(A\cap B)$.
Then $x\in A$ or $x\in A\cap B\subseteq A$, and in both cases we have $x\in A$.
Conversely, if $x\in A$, we clearly have $x\in A\cup(A\cap B)$.

\myem{3.}
Suppose that $x\in A\cup(B\cap C)$.
Then $x\in A$ or $x\in B\cap C$.
If $x\in A$, then $x\in A\cup B$ and $x\in A\cup C$.
Thus $x\in (A\cup B)\cap(A\cup C)$.
If $x\in B\cap C$, then $x\in B$ and $x\in C$.
Thus $x\in A\cup B$ and $x\in A\cup C$.
Therefore $x\in (A\cup B)\cap(A\cup C)$.

Conversely, suppose that $x\in (A\cup B)\cap(A\cup C)$.
Then $x\in A\cup B$ and $x\in A\cup C$.
If $x\in A$, then clearly $x\in A\cup(B\cap C)$.
If instead $x\not\in A$, we have $x\in B$ and $x\in C$.
Thus $x\in B\cap C$.
Therefore $x\in A\cup(B\cap C)$.
\end{proof}


\begin{hw}
Let $A$ and $B$ be sets.
Show that $A\subseteq B$ if and only if $A-B=\emptyset$.
\end{hw}


\begin{hw}
Let $A$, $B$, and $C$ be sets.
Prove the following properties.
\begin{myEnum}
\item $A-(B\cup C) = (A-B)\cap(A-C)$\hfill\myem{(de Morgan law for $\cup$)}
\item $A-(B\cap C) = (A-B)\cup(A-C)$\hfill\myem{(de Morgan law for $\cap$)}
\end{myEnum}
\end{hw}


\begin{defn}
\TIP{\centerline{$A^2=A\times A$}

\centerline{$A^k=\{(x_1,\ldots,x_k)\,|\,x_i\in A\}$}}
A \term{list} with entries in $A$ consists of elements of $A$ in a specified order,
with repetition allowed.
A \term{$k$-tuple} is a list with $k$ entries.
We write $A^k$ for the set of $k$-tuples with entries in $A$.
An \term{ordered pair} is a list with two entries.
The \term{Cartesian product} of sets $S$ and $T$, written $S\times T$,
is the set $\{(x,y)\,|\, x\in S, y\in T\}$.
Two ordered pairs $(a,b)$ and $(c,d)$ are \term{equal}, written $(a,b)=(c,d)$,
if $a=c$ and $b=d$.
\end{defn}

