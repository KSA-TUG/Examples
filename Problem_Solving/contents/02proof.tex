% !TEX root = ./ProblemSolving.tex

\chapter{Language of Mathematics}\label{chap:language}


%%%%%%%%%%%%%%%%%%%%%%%%%%%%%%%%%%%%%%%%%%%%%%%%%%%%%%%%%%%%%%%%%%%%%%%%%%%%%%%%%%%%%%%%%%%%
\section{Mathematical Statement}
%%%%%%%%%%%%%%%%%%%%%%%%%%%%%%%%%%%%%%%%%%%%%%%%%%%%%%%%%%%%%%%%%%%%%%%%%%%%%%%%%%%%%%%%%%%%
%%%%%%%%%%%%%%%%%%%%%%%%%%%%%%%%%%%%%%%%%%%%%%%%%%%%%%%%%%%%%%%%%%%%%%%%%%%%%%%%%%%%%%%%%%%%


A \term{proposition} or \term{mathematical statement} is a sentence which is either true or false, but not both.
Consider following list.
\begin{myEnum}
\item $\pi = 3$.
\item $1+1=2$.
\item $12$ may be written as the sum of two prime numbers.
\item The square of even integer is even.
\item Every even integer greater than $2$ may be written as the sum of two prime numbers.
\item $n$ is a prime number.
\item $n^2-2n>2$.
\item $m<n$
\item $12-11$.
\item $\pi$ is a special number.
\end{myEnum}

\TIP[-30pt]{The Goldbach conjecture is suggested by Christian Goldbach
and it might be true in a letter to Leonhard Euler written in 1742.}
The first five statements \mybf{1} to \mybf{5} are propositions.
In fact \mybf{1} is false and 
the statement \mybf{5} is not yet proved, called \term{Goldbach conjecture}.
Statements \mybf{2}, \mybf{3}, and \mybf{4} are true.

For other statements, we can not determine whether true or false.
Thus they are not propositions.
The statement \mybf{6} and \mybf{7} become propositions once a numerical \term{value} is assigned to $n$.
For example, if $n=2$, then \mybf{6} is true and \mybf{7} is false,
whereas if $n=3$ then they are both true.
Similarly, \mybf{8}  becomes a proposition when values are assigned to both $m$ and $n$.
Sentence of this type are called \term{predicates}.
The symbols which need to be given values in order to obtain a propositions are called \term{free variables}.

The word \term{statement} will be used to denote either a proposition or predicate.
So in the above, the first eight items are statements.
We will use a single capital letter $P$ or $Q$ to indicate a statement,
or sometimes an expression like $P(m,n)$ to indicate a predicate,
with the free variables listed.



%%%%%%%%%%%%%%%%%%%%%%%%%%%%%%%%%%%%%%%%%%%%%%%%%%%%%%%%%%%%%%%%%%%%%%%%%%%%%%%%%%%%%%%%%%%%
\section{Logical Connectives}
%%%%%%%%%%%%%%%%%%%%%%%%%%%%%%%%%%%%%%%%%%%%%%%%%%%%%%%%%%%%%%%%%%%%%%%%%%%%%%%%%%%%%%%%%%%%
%%%%%%%%%%%%%%%%%%%%%%%%%%%%%%%%%%%%%%%%%%%%%%%%%%%%%%%%%%%%%%%%%%%%%%%%%%%%%%%%%%%%%%%%%%%%
\subsection{Or}

Consider the following statement:
\begin{quotation}
For integers $a$ and $b$, $ab=0$ if $a=0$ or $b=0$.
\end{quotation}
\TIP{일상생활에서 사용하는 언어에서 \myem{or}는 \myem{exclusive or}의 뜻으로 쓰인다.
예를 들어 \myem{every student will take Pass or Fail on the research course.}라는 문장에서 or는
exclusive or의 뜻으로 Pass와 Fail을 동시에 받을 수 없음을 내포하고 있다.}
The statement \myem{$a=0$ or $b=0$} is true if $a=0$ (regardless of the value of $b$)
and is also true if $b=0$ (regardless of the value of $a$).
Notice that the statement is true of both $a=0$ and $b=0$ are true.
This called the \term{inclusive} use of \term{or}.
Any statement may be either true or false:
we say that \term{true} and \term{false} are the two possible \term{truth values}
for the statement.
So given two statements $P$ and $Q$ each has two possible truth values
giving four possible combinations in all.
The \term{truth table} for \myem{or} which follows specifies the truth value for
\myem{$P$ or $Q$} corresponding to each possible combination of truth value for $P$ and for $Q$,
one line for each.
In the table, $T$ indicates \myem{true} and $F$ indicates \myem{false}.
$$
\begin{tabular}{cc|c}
$P$ & $Q$ & $P$ or $Q$\\ \hline
$T$ & $T$ & $T$\\
$T$ & $F$ & $T$\\
$F$ & $T$ & $T$\\
$F$ & $F$ & $F$
\end{tabular}
$$
\myem{$P$ or $Q$} is called the \term{disjunction} of the two statements $P$ and $Q$.
And we use $\vee$ indicate disjunction, so $P\vee Q$ means $P$ or $Q$.


\begin{eg}
The connection \myem{or} is sometimes hidden in other notation.
\begin{myEnum}
\item $a\leq b$ means $a<b$ or $a=b$.
\item $a=\pm b$ means $a=b$ or $a=-b$.
\end{myEnum}
\end{eg}



%%%%%%%%%%%%%%%%%%%%%%%%%%%%%%%%%%%%%%%%%%%%%%%%%%%%%%%%%%%%%%%%%%%%%%%%%%%%%%%%%%%%%%%%%%%%
\subsection{And}

If we assert that
\begin{quotation}
$\pi$ lies between $3$ and $4$
\end{quotation}
or in symbols
\begin{quotation}
$3<\pi<4$
\end{quotation}
then this means that
\begin{quotation}
$\pi>3$ and $\pi<4$
\end{quotation}
which is called \term{conjunction} of the two statements \myem{$\pi>3$} and \myem{$\pi<4$}.
And we use $\wedge$ indicate conjunction, so $P\wedge Q$ means $P$ and $Q$.
We have the following truth table for \term{and}.
$$
\begin{tabular}{cc|c}
$P$ & $Q$ & $P$ and $Q$\\ \hline
$T$ & $T$ & $T$\\
$T$ & $F$ & $F$\\
$F$ & $T$ & $F$\\
$F$ & $F$ & $F$
\end{tabular}
$$



%%%%%%%%%%%%%%%%%%%%%%%%%%%%%%%%%%%%%%%%%%%%%%%%%%%%%%%%%%%%%%%%%%%%%%%%%%%%%%%%%%%%%%%%%%%%
\subsection{Not}
Finally we have the idea of the \term{negation} of a statement.
The negation of a statement is true when the original statement is false and
it is false when the original statement is true.
This described by the following truth table.
$$
\begin{tabular}{c|c}
$P$ & not $P$\\ \hline
$T$ & $F$\\
$F$ & $T$
\end{tabular}
$$


\TIP{일상에서 사용하는 언어에서는 어느 정도의 모호한 말을 하더라고 듣는 사람이
전후사정을 감안하여 이해한다.
그러나 수학에서는 모호함을 완전히 배재한 문장을 구사해야 한다.}
\begin{eg}
What is the negation of \myem{All students are male}?
Some would say, incorrectly, \myem{All student is not male}.
The correct negation is \myem{At least one student is not male}.
\end{eg}


%%%%%%%%%%%%%%%%%%%%%%%%%%%%%%%%%%%%%%%%%%%%%%%%%%%%%%%%%%%%%%%%%%%%%%%%%%%%%%%%%%%%%%%%%%%%
\section{Quantifiers and Logical Statements}
%%%%%%%%%%%%%%%%%%%%%%%%%%%%%%%%%%%%%%%%%%%%%%%%%%%%%%%%%%%%%%%%%%%%%%%%%%%%%%%%%%%%%%%%%%%%
%%%%%%%%%%%%%%%%%%%%%%%%%%%%%%%%%%%%%%%%%%%%%%%%%%%%%%%%%%%%%%%%%%%%%%%%%%%%%%%%%%%%%%%%%%%%
\subsection{Quantifiers}

\begin{defn}
In the statement
\begin{quotation}
For all $x$ in $S$, $P(x)$ is true.
\end{quotation}
the variable $x$ is \term{universally quantified}.
We write this as
\[
(\forall x\in S)\,P(x)
\]
and say that $\forall$ is a \term{universal quantifier}.
In
\begin{quotation}
There exists an $x$ in $S$ such that $P(x)$ is true.
\end{quotation}
the variable $x$ is \term{existentially quantified}.
We write this as
\[
(\exists x\in S)\,P(x)
\]
and say that $\exists$ is an \term{existential quantifier}.
The set of allowed values for a variable is its \term{universe}.
\end{defn}

Typically, \myem{every} and \myem{for all} represent universal quantifiers,
while \myem{some} and \myem{there is} represent existential quantifiers.
Following list is common indicators of quantification.
$$
\begin{tabular}{ll|ll}\hline
Universal ($\forall$) & (helpers) & Existential ($\exists$) & (helpers) \\ \hline
for (all), for every & & for some & \\
if & then & there exists & such that\\
whenever, for ,given & & at least one & for which\\
every, any & satisfies & some & satisfies\\
a, arbitrary & must, is & has a & such that\\
let & be & & \\ \hline
\end{tabular}
$$
Sometimes the \term{helper} may be absent.


%%%%%%%%%%%%%%%%%%%%%%%%%%%%%%%%%%%%%%%%%%%%%%%%%%%%%%%%%%%%%%%%%%%%%%%%%%%%%%%%%%%%%%%%%%%%
\subsection{Order of Quantifiers}

\begin{eg}
Consider the sentence
\begin{quote}
There is a real number $y$ such that $x=y^3$ for every real number $x$.
\end{quote}
It seems to say that some number $y$ is the cube root of all numbers,
which is false.
To say that every number has a cube root, we write
\begin{quote}
For every real number $x$, there is a real number $y$ such that $x=y^3$.
\end{quote}
\end{eg}


Compare the statements below.
\[
(\forall x\in A)(\exists y\in B)\, P(x,y) \qquad\qquad (\exists y\in B)(\forall x\in A)\, P(x,y)
\]
The first statement is true if for each $x$ we can take a $y$ that works.
For the second statement to be true,
there must be a single $y$ that will always work,
no matter which $x$ is chosen.

\begin{eg}
Let $A$ be the set of all children, let $B$ be the set of all parents,
and let $P(x,y)$ be \myem{$y$ is the parent of $x$}.
Then
\begin{myEnum}
\item $(\forall x\in A)(\exists y\in B)\, P(x,y)$:
For all children, there is a parent.
\item $(\exists y\in B)(\forall x\in A)\, P(x,y)$:
There is a parent who is the parent of all children.
\end{myEnum}
The first statement is true, but the second statement is to strong and is not true.
\end{eg}


\begin{eg}
Let $A=B=\mathbb{R}$, and let $P(x,y)$ be $xy=0$.
Then following statements are true.
\begin{myEnum}
\item $(\forall x\in A)(\exists y\in B)\, P(x,y)$:
For all $x\in\mathbb{R}$, there is a $y\in\mathbb{R}$ such that $xy=0$.
\item $(\exists y\in B)(\forall x\in A)\, P(x,y)$:
There is a $y\in\mathbb{R}$ such that $xy=0$ for all $x\in\mathbb{R}$.
\end{myEnum}
\end{eg}



%%%%%%%%%%%%%%%%%%%%%%%%%%%%%%%%%%%%%%%%%%%%%%%%%%%%%%%%%%%%%%%%%%%%%%%%%%%%%%%%%%%%%%%%%%%%
\subsection{Negation of Quantified Statements}

After placing a statement involving quantifiers in the conventional order,
negating the statement is easy.
If it is false that $P(x)$ is true for every value of $x$,
then there must be some value of $x$ such that $P(x)$ is false,
and vise versa.
Similarly, if it is false that $P(x)$ is true for some value of $x$,
then $P(x)$ is false for every value of $x$.
Thus in notatin,
\begin{quotation}
$\neg[(\forall x)\,P(x)]$ has the same meaning as $(\exists x)(\neg P(x))$.

$\neg[(\exists x)\,P(x)]$ has the same meaning as $(\forall x)(\neg P(x))$.
\end{quotation}
Note that when using logical symbols, we may add matched parentheses or brackets to clarify grouping.


\begin{eg}
The negation of \myem{Every good boy does fine} is \myem{Some good boy does not do fine};
it says noting about bad boys.
Similarly, the negation of \myem{Every chair in this room is broken} is \myem{Some chair in this room is not broken};
it says nothing about chairs outside this room.
\end{eg}



\begin{eg}
Negate the statement $(\forall n\in\mathbb{N})(\exists x\in A)(nx<1)$.
\end{eg}
\begin{sol}
The negation is $(\exists n\in\mathbb{N})(\forall x\in A)(nx\ge 1)$.
This means that the set $A$ has a lower bound that is the reciprocal of an integer.
It does not mention values of $n$ outside $\mathbb{N}$ or values of $x$ outside $A$.
\end{sol}


\begin{eg}
Rephrase the following statement so that it is clear.

\centerline{\myem{It is false that every classroom has a chair that is not broken.}}
\end{eg}
\begin{sol}
The qualifiers make it improper to cancel the \myem{double negative};
the sentence \myem{every classroom has a chair that is broken} has a different meaning.

Let $R$ denote the set of classrooms.
Given a room $r$, let $C(r)$ denote the set of chairs in $r$.
For a chair $c$, let $B(c)$ be the statement that $c$ is broken.
Then
\begin{align*}
& \neg\left[(\forall r\in R)(\exists c\in C(r))(\neg B(c))\right]\tag{1}\\
\Longrightarrow \ & (\exists r \in R)\left(\neg\left[(\exists c\in C(r))(\neg B(c))\right]\right)\tag{2}\\
\Longrightarrow \ & (\exists r \in R)(\forall c\in C(r))\, B(c)\tag{3}\\
\end{align*}
Each symbolic statement means
\begin{myEnum}
\item It is false that every classroom has a chair that is not broken.
\item There is a classroom that has no chair that is not broken.
\item There is a classroom in which every chair is broken.
\end{myEnum}
Therefore the given statement mean that

\centerline{\myem{There is a classroom in which every chair is broken.}}
\end{sol}


%%\begin{eg}
%%Negate the statement every classroom has a chair that is not broken.
%%\end{eg}
%%\begin{sol}
%%The negation is
%%
%%\centerline{\myem{It is false that every classroom has a chair that is not broken.}}
%%
%%The qualifiers make it improper to cancel the \myem{double negative};
%%the sentence \myem{every classroom has a chair that is broken} has a different meaning.
%%
%%Let $R$ denote the set of classrooms.
%%Given a room $r$, let $C(r)$ denote the set of chairs in $r$.
%%For a chair $c$, let $B(c)$ be the statement that $c$ is broken.
%%Then
%%\begin{align*}
%%& \neg\left[(\forall r\in R)(\exists c\in C(r))(\neg B(c))\right]\tag{1}\\
%%\Longrightarrow \ & (\exists r \in R)\left(\neg\left[(\exists c\in C(r))(\neg B(c))\right]\right)\tag{2}\\
%%\Longrightarrow \ & (\exists r \in R)(\forall c\in C(r))\, B(c)\tag{3}\\
%%\end{align*}
%%Each symbolic statement means
%%\begin{myEnum}
%%\item It is false that every classroom has a chair that is not broken.
%%\item There is a classroom that has no chair that is not broken.
%%\item There is a classroom in which every chair is broken.
%%\end{myEnum}
%%Therefore the negation is \myem{There is a classroom in which every chair is broken}.
%%\end{sol}




%%%%%%%%%%%%%%%%%%%%%%%%%%%%%%%%%%%%%%%%%%%%%%%%%%%%%%%%%%%%%%%%%%%%%%%%%%%%%%%%%%%%%%%%%%%%
\section{Compound Statements}
%%%%%%%%%%%%%%%%%%%%%%%%%%%%%%%%%%%%%%%%%%%%%%%%%%%%%%%%%%%%%%%%%%%%%%%%%%%%%%%%%%%%%%%%%%%%
%%%%%%%%%%%%%%%%%%%%%%%%%%%%%%%%%%%%%%%%%%%%%%%%%%%%%%%%%%%%%%%%%%%%%%%%%%%%%%%%%%%%%%%%%%%%

\begin{defn}
Let $P$ and $Q$ be statements.
The logical connective \term{conditional}, written by $P\Rightarrow Q$, means that
$P$ implies $Q$.
And \term{biconditional}, written by $P\Leftrightarrow Q$, means that
$P$ if and only if $Q$.

In the conditional statement $P\Rightarrow Q$, we call $P$ the \term{hypothesis} and
$Q$ the \term{conclusion}.
The statement $Q\Rightarrow P$ is the converse of $P\Rightarrow Q$.
\end{defn}


Conditional statements are the only type $P\implies Q$ whose meaning changes when $P$ and $Q$ are interchanged.
There is no general relationship between the truth values of $P\implies Q$ and $Q\implies P$.


\begin{eg}
Let $x$ be a real number, and $P$ be the statement $x>0$, $Q$ be $x^2>0$, and $R$ be $x+1>1$.
Here $P\implies Q$ is true but $Q\implies P$ is false.
On the other hand, both $P\implies R$ and $R\implies P$ are true.
\end{eg}

It may be helpful to read the conditional as \myem{if-then} instead of \myem{implies}.
The following list show that the way to say $P\implies Q$ in English.
$$
\begin{tabular}{llll}
If $P$ (is true), then $Q$ (is true). & & $P$ is true only if $Q$ is true.\\
$Q$ is true whenever $P$ is true. & & $P$ is a sufficient condition for $Q$.\\
$Q$ is true if $P$ is true. & & $Q$ is a necessary condition for $P$.
\end{tabular}
$$


The mathematician defines a truth table for $P\implies Q$ just as he does for
$\neg$, $\vee$, and $\wedge$.
But the definition is not at all obvious.
The following example may help before we give the truth table.
Consider the sentence
\begin{quotation}
If I get A+ in mathematics, then I will serve pizza for all students.
\end{quotation}
Suppose a student says this.
When is he telling the truth and when is he lying?
Examine the following four cases where $P$ means \myem{I get an A+ in mathematics}
and $Q$ means \myem{I will serve pizza for all students}.
\begin{myEnum}
\item $P$ is true: He get an A+ in mathematics\\
$Q$ is true: He serves pizza for all students

\item $P$ is true: He get an A+ in mathematics\\
$Q$ is false: He does not serve pizza for all students

\item $P$ is false: He does not get an A+ in mathematics\\
$Q$ is true: He serves pizza for all students

\item $P$ is false: He does not get an A+ in mathematics\\
$Q$ is false: He does not serve pizza for all students
\end{myEnum}
In case \mybf{1}, it is reasonable to agree that the student was telling the truth; his claim is true.
In case \mybf{2}, it is easy to agree that he lied, and his claim was false.
In case \mybf{3}, you could not call him a liar since he serves pizza even though he did not get an A+.
In case \mybf{4}, you likewise could not call him a liar since he did not get the A+ and did not serve pizza.
Thus we have following truth table for $P\implies Q$.
$$
\begin{tabular}{cc|c}
$P$ & $Q$ & $P\implies Q$\\ \hline
$T$ & $T$ & $T$\\
$T$ & $F$ & $F$\\
$F$ & $T$ & $T$\\
$F$ & $F$ & $T$
\end{tabular}
$$
And actually, $P\iff Q$ is true whenever $P$ and $Q$ have same truth value.
Thus we have following truth table.
$$
\begin{tabular}{cc|c}
$P$ & $Q$ & $P\iff Q$\\ \hline
$T$ & $T$ & $T$\\
$T$ & $F$ & $F$\\
$F$ & $T$ & $F$\\
$F$ & $F$ & $T$
\end{tabular}
$$


%%%%%%%%%%%%%%%%%%%%%%%%%%%%%%%%%%%%%%%%%%%%%%%%%%%%%%%%%%%%%%%%%%%%%%%%%%%%%%%%%%%%%%%%%%%%
\subsection{Logical Equivalences}

\begin{defn}
Two logical expression $X$, $Y$ are \term{logically equivalent}
if they have the same truth value for each assignment of truth values to the variables.
And a \term{tautology} is a logical expression which is true no matter what the truth values of its variables.
\end{defn}


\begin{eg}
Show that $P\implies Q$ and $(\neg P)\vee Q$ are logically equivalent,
and so the expression $R$ given by $(P\implies Q)\iff((\neg P)\vee Q)$ is a tautology.
\end{eg}
\begin{sol}
Consider truth tables;
$$
\begin{tabular}{cc|c|c|c|c}
$P$ & $Q$ & $P\implies Q$ & $\neg P$ & $(\neg P)\vee Q$ & R\\ \hline
$T$ & $T$ & $T$ & $F$ & $T$ & $T$\\
$T$ & $F$ & $F$ & $F$ & $F$ & $T$\\
$F$ & $T$ & $T$ & $T$ & $T$ & $T$\\
$F$ & $F$ & $T$ & $T$ & $T$ & $T$
\end{tabular}
$$
Thus they are logically equivalent.
So $R$ is a tautology.
\end{sol}


We may substitute $P$ for $\neg(\neg P)$ whenever we wish, and vise versa.
Similarly, $P\vee Q$ is equivalent to $Q\vee P$, and $P\wedge Q$ is equivalent to $Q\wedge P$.

\begin{eg}[Elementary logical Equivalences]
Whenever $P$ and $Q$ are statements, we have following logical equivalences
$$
\begin{tabular}{llcl}
\mybf{1.} & $\neg(P\wedge Q)$ & $\iff$ & $(\neg P)\vee (\neg Q)$\\
\mybf{2.} & $\neg(P\vee Q)$ & $\iff$ & $(\neg P)\wedge (\neg Q)$\\
\mybf{3.} & $\neg(P\implies Q)$ & $\iff$ & $P\wedge (\neg Q)$\\
\mybf{4.} & $P\iff Q$ & $\iff$ & $(P \implies Q)\wedge (Q \implies P)$\\
\mybf{5.} & $P\vee Q$ & $\iff$ & $(\neg P)\implies Q$\\
\mybf{6.} & $P\implies Q$ & $\iff$ & $(\neg Q)\implies (\neg P)$\\
\end{tabular}
$$
Equivalences \mybf{1} and \mybf{2} are called \term{de Morgan's laws}.

Equivalences \mybf{3} and \mybf{4} restate the definitions of the conditional and biconditional.
A conditional statement is false precisely when the hypothesis is true and the conclusion is false.
The biconditional is true precisely when the conditional and its converse are both true.

Each side of \mybf{5} is false precisely when $P$ fails and $Q$ fails.
Each side of \mybf{6} is false precisely when $P$ true and $Q$ fails.
\end{eg}



%%%%%%%%%%%%%%%%%%%%%%%%%%%%%%%%%%%%%%%%%%%%%%%%%%%%%%%%%%%%%%%%%%%%%%%%%%%%%%%%%%%%%%%%%%%%
\subsection{Logical Connectives and Membership in Sets}
Let $P(x)$ and $Q(x)$ be statements about an element $x$ from a universe $U$.
Often we write a conditional statement $(\forall x\in U)(P(x)\implies Q(x))$
as $P(x)\implies Q(x)$ or simply $P\implies Q$ with an implicit universal quantifier.

The hypothesis $P(x)$ can be interpreted as a universal quantifier in another way.
With $A=\{x\in U\,|\, \text{$P(x)$ is true}\}$, the statement $P(x)\implies Q(x)$ can be
written as $(\forall x\in A)Q(x)$.

Another interpretation of $P(x)\implies Q(x)$ uses set inclusion.
With $B=\{x\in U\,|\, \text{$Q(x)$ is true}\}$, the conditional statement has the same meaning as
the statement $A\subseteq B$.
The converse statement $Q(x)\implies P(x)$ is $B\subseteq A$.
Thus the biconditional $P\iff Q$ is equivalent to $A=B$.


\begin{eg}
Let $P$ be the statement of membership in $A$ and
$Q$ be the statement of membership in $B$.
Then we have following equivalences.
$$
\begin{tabular}{lcccccc}
$x\in A^c$ & $\iff$ & not $(x\in A)$ & $\iff$ & $\neg (x\in A)$ & $\iff$ & $\neg P$\\
$x\in A\cup B$ & $\iff$ & $(x\in A)$ or $(x\in B)$ & $\iff$ & $(x\in A)\vee(x\in B)$ & $\iff$ & $P\vee Q$\\
$x\in A\cap B$ & $\iff$ & $(x\in A)$ and $(x\in B)$ & $\iff$ & $(x\in A)\wedge(x\in B)$ & $\iff$ & $P\wedge Q$\\
$A\subseteq B$ & $\iff$ & $(\forall x\in A)(x\in B)$ & $\iff$ & $(x\in A)\implies(x\in B)$ & $\iff$ & $P\implies Q$
\end{tabular}
$$
\end{eg}


In the language of sets, de Morgan's laws became
\[
(A\cap B)^c = A^c \cup B^c\tag{*}
\]
We verify \mybf{*} by translation into a logical equivalence about membership.
Given an element $x$,
Let $P$ be the property $x\in A$, and let $Q$ be the property $x\in B$.
Then
\begin{align*}
x\in (A\cap B)^c & \iff \neg(P\wedge Q)\\
& \iff (\neg P)\vee(\neg Q)\\
& \iff (x\not\in A)\vee(x\not\in B)\\
& \iff (x\in A^c)\vee(x\in B^c)\\
& \iff x\in A^c\cup B^c
\end{align*}
Alternatively, a Vann diagram makes the reasoning clear.


\begin{hw}
Verify the de Morgan's laws $(A\cup B)^c = A^c \cap B^c$ using logical equivalence about membership.
\end{hw}



%%%%%%%%%%%%%%%%%%%%%%%%%%%%%%%%%%%%%%%%%%%%%%%%%%%%%%%%%%%%%%%%%%%%%%%%%%%%%%%%%%%%%%%%%%%%
\section{Proofs}
%%%%%%%%%%%%%%%%%%%%%%%%%%%%%%%%%%%%%%%%%%%%%%%%%%%%%%%%%%%%%%%%%%%%%%%%%%%%%%%%%%%%%%%%%%%%
%%%%%%%%%%%%%%%%%%%%%%%%%%%%%%%%%%%%%%%%%%%%%%%%%%%%%%%%%%%%%%%%%%%%%%%%%%%%%%%%%%%%%%%%%%%%

A proof of a mathematical statement is a logical argument which shows the truth of the statement.
The logical argument consists of several steps provided by implications.
In this chapter, we will describe a variety of methods of proof
so that students can apply to proving a mathematical statement.

%%%%%%%%%%%%%%%%%%%%%%%%%%%%%%%%%%%%%%%%%%%%%%%%%%%%%%%%%%%%%%%%%%%%%%%%%%%%%%%%%%%%%%%%%%%%
\subsection{Direct Proofs}

Most of theorems are of the form $P\implies Q$.
Since the statement is necessarily true if $P$ is false,
we only need consider the case when $P$ is true.
Then form the truth table we know that $P\implies Q$ is true whenever $Q$ is also true.

Thus to prove that $P\implies Q$ is true,
it is sufficient to assume that $P$ is true and deduce $Q$ is true by logical arguments.
This is the \term{direct proof}.


\begin{eg}\label{eg:dproof}
Show that for positive real numbers $a$ and $b$, $a<b\implies a^2<b^2$.
\end{eg}
\begin{sol}
The statement
\begin{quotation}
For positive real numbers $a$ and $b$, $a<b\implies a^2<b^2$.
\end{quotation}
consists of
\begin{myEnum}
\item quantifier: for all positive real number $a$ and $b$
\item hypothesis: $a<b$
\item conclusion: $a^2<b^2$
\end{myEnum}
We will deduce the conclusion from quantifiers and hypothesis using logical arguments as follows;

Since $a$ is a positive real number, we have
\begin{align*}
a< b & \implies a^2 < ab\tag{multiplying through by $a>0$}
\intertext{And since $b$ is also a positive,}
a< b & \implies ab < b^2\tag{multiplying through by $b>0$}
\intertext{Thus we have}
a< b & \implies (a^2 < ab)\ \text{and}\ (ab < b^2)
\end{align*}
Therefore if $a<b$, then $a^2<b^2$.
\end{sol}

This example highlights the fact that in constructing a proof as a chain of implications
we repeatedly use
\[
[(P\implies Q)\ \text{and}\ (Q\implies R)]\implies (P\implies R)
\]
which can be proved by using a truth table.


When proving universal implications it is often difficult to consider all the objects
satisfying the hypothesis.
Then we usually divide the hypothesis to several cases.

\begin{eg}[Proof by Cases]\label{eg:cases}
Prove that $a^2>0$ for non-zero real number $a$.
\end{eg}
\begin{sol}
Here
\begin{myEnum}
\item quantifier: for non-zero real number $a$
\item conclusion: $a^2>0$
\end{myEnum}
From the quantifier, we have
\begin{align*}
a\neq 0 & \implies a>0\ \text{or}\ a<0\tag{①}
\intertext{However, by the multiplication law,}
a > 0 & \implies a^2 >0
\intertext{and}
a < 0 & \implies a^2 >0
\intertext{Hence}
a>0\ \text{or}\ a<0 & \implies a^2>0\tag{②}
\end{align*}
Therefore, putting together statement ① and ② we have $a\neq 0\implies a^2>0$.
\end{sol}



\begin{eg}[Constructing Proofs Backwards]
Prove that $a<b\implies 4ab<(a+b)^2$ for real numbers $a$ and $b$.
\end{eg}
\begin{sol}
Here
\begin{myEnum}
\item quantifier: for non-zero real number $a$
\item hyphothesis: $a<b$
\item conclusion: $4ab<(a+b)^2$
\end{myEnum}
A difficulty of proof is that the conclusion is more complicated than hyphothesis
and it is not immediately clear how to reach this more complicated statement.
In such situation, it is often best to start with the more complicated statement
and to simplify.
In this case, simplifying the conclusion leads to the construction of a proof backwards.
\begin{align*}
4ab < (a+b)^2 & \Leftarrow 4ab < a^2 + 2ab + b^2\\
& \Leftarrow 0 < a^2 - 2ab + b^2\\
& \Leftarrow 0 < (a - b)^2\\
& \Leftarrow 0 \neq a - b\tag{by Example~\ref{eg:cases}}\\
& \Leftarrow a\neq b\\
& \Leftarrow a < b
\end{align*}
Thus $a<b\implies 4ab<(a+b)^2$.
\end{sol}


%%%%%%%%%%%%%%%%%%%%%%%%%%%%%%%%%%%%%%%%%%%%%%%%%%%%%%%%%%%%%%%%%%%%%%%%%%%%%%%%%%%%%%%%%%%%
\subsection{Contrapositive}
\TIP{Contrapositive = 대우}

The direct method can be inconvenient and does not always work.
In this section we will consider a logically equivalent
but very common and useful method of proof.

The \term{contrapositive} of $P\implies Q$ is $\neg Q\implies \neg P$.
The equivalence between a conditional and its contrapositive allows us
to prove $P\implies Q$ by proving $\neg Q\implies \neg P$.
This is the \term{contrapositive method}.


\begin{eg}\label{eg:count}
Consider the statement
\begin{quotation}
Let $f(x)=mx+b$. If $x\neq y$, then $f(x)\neq f(y)$.\hfill\mybf{*}
\end{quotation}
The direct method considers $x<y$ and $x>y$ separately and obtain $f(x)<f(y)$ or $f(x)>f(y)$.
This unsatisfying analysis by cases results from \myem{not equals} being a messier condition than \myem{equals}.

We can use the contrapositive to retain the language of equalities and reduce analysis by cases.
The contrapositive of the statement \mybf{*} is
\begin{quotation}
Let $f(x)=mx+b$. If $f(x) = f(y)$, then $x = y$. 
\end{quotation}
When $f(x)=f(y)$, we obtain $mx+b = my +b$ and then $mx=my$.
If $m\neq 0$, then we obtain $x=y$.

If $m=0$, then we cannot divide by $m$, and actually the statement is false.
Thus the statement is true if and only if $m\neq 0$.
\end{eg}


A universally quantified statement like \myem{$(\forall x\in U)[P(x)\implies Q(x)]$}
can be disproved by finding an element $x$ in $U$ such that $P(x)$ is true and $Q(x)$ is false.
Such an element $x$ is a \term{counterexample}.
In Example~\ref{eg:count}, $m=0$ is a conuterexample to a claim that the implication holds for all $m$.


\begin{eg}
Prove that if $a$ is less than or equal to every real number greater than $b$, then $a\leq b$.
\end{eg}
\begin{sol}
The direct method goes nowhere,
but when we say \myem{suppose not}, the light begins to dawn.
If $a>b$, then $a>\frac{a+b}{2}>b$.
Thus $a$ is not less then or equal to every number greater than $b$.
We have proved the contrapositive of the desired statement.
\end{sol}



%%%%%%%%%%%%%%%%%%%%%%%%%%%%%%%%%%%%%%%%%%%%%%%%%%%%%%%%%%%%%%%%%%%%%%%%%%%%%%%%%%%%%%%%%%%%
\subsection{Indirect Proof}

\TIP{Method of Contradiction = 귀류법}
After negating both sides of $(P\implies Q) \iff [(\neg P)\vee Q]$, we have
\[ 
\neg (P\implies Q) \iff [P\wedge (\neg Q)].
\]
Thus
\[ 
(P\implies Q) \iff \neg [P\wedge (\neg Q)].
\]
Hence we can prove $P\implies Q$ by proving $P$ and $\neg Q$ cannot both be true.
We do this by obtaining a contradiction after assuming both $P$ and $\neg Q$.
This is the \term{method of contradiction} or \term{indirect proof}.

It may be useful to present a template for writing out these proofs.
Suppose that we are writing out a formal proof of the statement $P$
using the method of contraction.
We would set out the proof as follows.

\HintBox{Template for Method of Contradiction}{
Suppose, for contradiction, that the statement $Q$ is false.

Then [present some argument which leads to a contradiction of some sort].

Hence our assumption that $Q$ is false must be false.

Thus $Q$ is true as required.
}

\begin{eg}
Show that among the numbers $y_1$, $\ldots$, $y_n$, some number is as large as the average.
\end{eg}
\TIP{Let $Y=y_1+\cdots+y_n$, then the \term{average} $z$ is $Y/n$.}
\begin{sol}
An indirect proof of the claim begins,
\myem{suppose that the conclusion is false}.
Thus $y_i < z$ for all $y_i$ in the list.
If we sum these inequalities, we obtain
$Y < nz$,
but this contradicts the definition of $z$, which yields
$Y=nz$.
Here the assumption that each element is too small must be false.

A direct proof constructs the desired number.
Let $y^*$ be the largest number in the set.
We prove that this candidate is as large as the average.
Since $y_i \leq y^*$, we sum the inequalities to obtain $Y<ny^*$
and then divide by $n$ to obtain $z\leq y^*$.
\end{sol}



\begin{eg}
Show that there is no largest real number.
\end{eg}
\begin{sol}
If there is a largest real number $z$, then for all $x\in\mathbb{R}$, we have $z\geq x$.
When $x$ is the real number $z+1$, this yields $z\geq z+1$.
Subtracting $z$ from both sides yields $0\geq 1$.
This is a contradiction, and thus there is no largest real number.
\end{sol}



%%%%%%%%%%%%%%%%%%%%%%%%%%%%%%%%%%%%%%%%%%%%%%%%%%%%%%%%%%%%%%%%%%%%%%%%%%%%%%%%%%%%%%%%%%%%
\section{Exercises for Chapter~\ref{chap:sets} to \ref{chap:language}}
%%%%%%%%%%%%%%%%%%%%%%%%%%%%%%%%%%%%%%%%%%%%%%%%%%%%%%%%%%%%%%%%%%%%%%%%%%%%%%%%%%%%%%%%%%%%
%%%%%%%%%%%%%%%%%%%%%%%%%%%%%%%%%%%%%%%%%%%%%%%%%%%%%%%%%%%%%%%%%%%%%%%%%%%%%%%%%%%%%%%%%%%%

\TIP{$(a,b)=\{x\in\mathbb{R}\,|\, a<x<b\}$

$[c,d]=\{x\in\mathbb{R}\,|\, c\leq x\leq d\}$
}
\begin{hw}
Prove that, if $a\leq b$, then $[a,b]\subseteq (c,d)$ if and only if $c<a$ and $b<d$.
\end{hw}

\begin{hw}
Prove that, if $A\cap B \subseteq C$ and $x\in B$, then $x\not\in A-C$.
\end{hw}

\begin{hw}
Prove that, for subsets of a universal set $U$, $A\subseteq B$ if and only if $B^c\subseteq A^c$.
\end{hw}

\begin{hw}
Prove by contradiction that there do not exist integer $m$ and $n$ such that $14m+21n=100$.
\end{hw}

\begin{hw}
Prove by contradiction that for any integer $n$, $n^2$ is odd implies $n$ is odd.
\end{hw}

\begin{hw}
Prove that if $a$ is a real number such that $a^2\geq 7a$, then $a\leq 0$ or $a\geq 7$.
\end{hw}

\begin{hw}
Prove that, for all real number $a$ and $b$,
\begin{myEnum}
\item $|a|<|b|$ if and only if $a^2<b^2$.
\item $|a|=|b|$ if and only if $a^2=b^2$.
\end{myEnum}
\end{hw}

\begin{hw}
Prove that, for all real number $a$ and $b$, $|a+b|\leq |a|+|b|$.
Give a necessary and sufficient condition for equality.
\end{hw}


